\section{Hosts}
Wenn ein Netzwerk keine weiteren Subnetze enthält können Sie die Hosts
konfigurieren. Im folgenden werden Ihre Möglichkeiten dazu gezeigt.
Falls Sie die beschriebenen Kontextmenüeinträge nicht sehen können, stellen
Sie sicher, dass dem Netzwerk keine weiteren Subnetze zugewiesen sind.

\subsection{Hosts hinzufügen}
\label{HostsHinzufuegen}
Um Hosts zu einem Netzwerk hinzuzufügen haben Sie 3 Möglichkeiten.

\textit{Falls Sie für das Netzwerk bereits IPv6 konfiguriert haben, können Sie
beim hinzufügen von Hosts automatisiert IPv6 Adressen zuweisen. Dabei handelt
es sich um zufällig generierte Adressen nach EUI Standard. Bestätigen Sie dafür
die entsprechende Sicherheitsabfrage.}

\subsubsection{Einzelnen Host hinzufügen}
Sie können einzelne Hosts mit oder ohne einer spezifischen IPv4 Adresse hinzufügen.

\subsubsection{Automatisch}
Wählen Sie im Kontextmenü des Netzwerkes den Menüpunkt "`Host hinzufügen"'.
Ein neuer Host mit der kleinstmöglichen IPv4 Adresse wird dem Netzwerk
hinzugefügt.

\subsubsection{Mit spezifischer IPv4 Adresse}
Um einen spezifischen Host hinzuzufügen wählen Sie im Kontextmenü des Netzwerkes
den Menüpunkt "`Host mit IPv4 Adresse hinzufügen"'.
Geben Sie nun die IPv4 Adresse an und bestätigen Sie diese.

\subsection{Alle Hosts hinzufügen}
Sie können ein Netzwerk mit allen möglichen Hosts füllen. Wählen Sie dazu den
Menüpunkt "`Alle Hosts hinzufügen"' im Kontextmenü des Netzwerkes aus.
Bereits konfiguriert Hosts bleiben erhalten.

\subsection{Hosts entfernen}
Falls Sie einen Host entfernen, oder dem Netzwerk Subnetze hinzufügen wollen,
befolgen Sie einen der nachfolgenden Schritte.

\subsubsection{Alle Hosts entfernen}
\label{AlleHostsEntfernen}
Wählen Sie den Menüpunkt "`Alle Hosts entfernen"' im Kontextmenü eines Netzwerkes
um alle Hosts des Netzwerkes zu entfernen. Nach dieser Operation, können Sie
dem Netzwerk wieder Subnetze hinzufügen.

\subsubsection{Einzelnen Host entfernen}
Um einen einzelnen Host zu entfernen klicken Sie mit der rechten Maustaste auf
den Host in der Host Tabelle um das Kontextmenü zu öffnen. Wählen sie den
Menüpunkt "`Host entfernen"' und bestätigen Sie die Sicherheitsabfrage.

Sie können diese Operation auch auf mehrer Hosts gleichzeitig ausüben. Markieren
Sie die Hosts, die gelöscht werden sollen, und wählen Sie den entsprechenden
Eintrag im Kontextmenü.

\subsection{Host bearbeiten}
\label{HostBearbeiten}
Sie können die IPv6 Adresse und den Namen eines Hosts bearbeiten. Gehen Sie
dafür wie folgt vor:\\
Klicken Sie doppelt auf die Zelle in der Host Tabelle, die
Sie ändern wollen, oder stellen Sie sicher, dass die Zelle, die Sie ändern
wollen im Fokus liegt und tippen Sie den neuen Wert ein.

\textbf{Beachten Sie, dass Sie die IPv4 Adresse eines Hosts nicht bearbeiten können.
Um die IPv4 Adresse eines Host zu ändern, fügen Sie einen neuen Host mit der
neuen IPv4 Adresse hinzu und übertragen Sie die Daten.}