\section{Installation}
Befolgen Sie die folgenden Schritte um die Software \subnetcalc\ zu installieren.

\subsection{Voraussetzungen}

\subsubsection{Java}
Stellen Sie sicher, dass Sie Java auf Ihrem Gerät installiert haben. Für die
Ausführung von \subnetcalc\ ist mindestens die Version 1.8 in 32 oder 64 bit erforderlich.

\subsubsection{Betriebssystem}
Auf den folgenden Betriebssystemen wurde die Software erfolgreich getestet:
\begin{itemize}
    \item Windows 10
    \item Ubuntu 16.04
\end{itemize}

\textbf{Beachten Sie, dass die Software evtl. auch auf anderen Betriebssystemen
ordnungsgemäß funktioniert, jedoch kann dies nur für die oben genannten
garantiert werden.}

\subsection{Ausführung}
Um die Software auszuführen, befolgen Sie je nach Betriebssystem die folgenden
Schritte:

\subsubsection{Windows}
Um die Software zu starten öffnen Sie die Netzwerkplaner.jar Datei, die Sie
erhalten haben mit einem Doppelklick.

Alternativ navigieren Sie per Kommandozeile in den Ordner, der die
Netzwerkplaner.jar Datei enthält und führen Sie folgenden Befehl aus:

\begin{lstlisting}
    java -jar .\Netzwerkplaner.jar
\end{lstlisting}

\subsubsection{Ubuntu}
Verfahren Sie, wie für die Windows-Version, indem Sie im Dateimanager
die Netzwerkplaner.jar Datei per Doppelklick öffnen.

Alternativ navigieren Sie per Shell in den Ordner, der die
Netzwerkplaner.jar Datei enthält und führen Sie folgenden Befehl aus:

\begin{lstlisting}
    java -jar ./Netzwerkplaner.jar
\end{lstlisting}