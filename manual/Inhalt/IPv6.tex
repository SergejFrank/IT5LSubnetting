\section{IPv6}
Der \subnetcalc\ unterstützt das Internet Protokoll in der Version 6.

\subsection{Globaler IPv6 Präfix}
Bei der Planung wird ein globaler IPv6 Präfix genutzt. Dieser ist optional und
kann z.B. einen Präfix repräsentieren, den Sie von Ihrem Internet Provider
erhalten haben.
Beim Start der Software ist der globale IPv6 Präfix auf \ipAddress{2001:db8::/32}
gesetzt. 

\subsubsection{Globalen Präfix anlegen}
\label{IPv6GlobalZuweisen}
Den globalen IPv6 Präfix können Sie anlegen, oder bearbeiten indem Sie das
Kontextmenü des Stammeintrages "`Netzwerke"' aufrufen und den Menüpunkt
"`Globalen IPv6 Präfix hinzufügen"' oder "`Globalen IPv6 Präfix bearbeiten"'
auswählen. geben Sie eine gültige IPv6 Netzwerk ID sowie eine IPv6 Präfixlänge
an. Stellen Sie sicher, dass alle IPv6 konfigurierten Netzwerke im
angegebenen IPv6 Präfix liegen.
Beispielhaft wären folgende Eingaben gültig:\\
\ipAddress{2001:db8:3b::/48}\\
\ipAddress{c11:29ad:0:49bf::/64}

\subsubsection{Globalen Präfix entfernen}
Um den Globalen IPv6 Präfix zu entfernen, rufen Sie das Kontextmenü des
Stammeintrages auf und wählen Sie den Menüpunkt "`Globalen IPv6 Präfix entfernen"'.
Alle IPv6 konfigurierten Netzwerke behalten ihre IPv6 Konfiguration.

\subsection{IPv6 in Netzwerken}
Um Netzwerke mit IPv6 zu konfigurieren gehen Sie wie folgt vor:

\subsubsection{IPv6 für Netzwerk konfigurieren}
Stellen Sie zunächst sicher, dass ein evtl. vorhandenes übergeordnetes Netzwerk
eine IPv6 Konfiguration besitzt. Anschließend wählen Sie den Menüpunkt
"`IPv6 zuweisen"' order "`IPv6 bearbeiten"' im Kontextmenü des Netzwerkes und
fahren Sie wie in \ref{IPv6GlobalZuweisen} beschrieben fort.

Sollte das Netzwerk bereits Hosts besitzen, können Sie allen Hosts eine zufällige
IPv6 Adresse zuweisen. Bestätigen Sie dafür die entsprechende Abfrage mit \key{Ja}.

\subsubsection{IPv6 aus Netzwerk entfernen}
Wenn Sie die IPv6 Konfiguration von einem Netzwerk entfernen wollen,
wählen sie den Menüpunkt "`IPv6 entfernen"' im Kontextmenü des Netzwerkes.
Beachten Sie, dass anders als beim globalen IPv6 Präfix auch alle Subnetzwerke
und Hosts ihre IPv6 Konfiguration verlieren.

\subsection{IPv6 Adressen fpr Hosts}
Um Hosts für IPv6 zu konfigurieren, muss zuerst das Netzwerk eine IPv6 Konfiguration besitzen.

\subsubsection{IPv6 Adresse zu Hosts hinzufügen}
\label{HostIPv6Hinzufuegen}
Wie bereits in \ref{HostBearbeiten} beschrieben, können Sie die IPv6 Adresse eines Hosts
durch die Hosttabelle ändern.
Stellen Sie sicher, dass eine eingegebene IPv6 Adresse im Bereich des Netzwerkes liegt und
noch nicht vergeben ist.

Alternativ können Sie auch für einen order mehrere Hosts automatisiert IPv6 Adressen
vergeben lassen. Markieren Sie dafür die Hosts in der Tabelle und wählen Sie im Kontextmenü
den Menüpunkt "`Zufällige IPv6 Adresse zuweisen"'.

\subsubsection{IPv6 Adresse von Host entfernen}
Um die IPv6 Adresse von einem Host zu entfernen, gehen Sie vor wie in
\ref{HostIPv6Hinzufuegen} beschrieben und lassen Sie die Tabellenzelle leer.