\section{Netzwerke}
Im folgenden wird der Umgang mit Netzwerken und ihrer Unterteilung in Subnetze
beschrieben.
Falls Sie die beschriebenen Kontextmenüeinträge nicht sehen können, stellen Sie sicher,
dass das Netzwerk noch keine Hosts enthält. Falls dies doch der Fall ist, entfernen Sie
diese, wie in \ref{AlleHostsEntfernen} beschrieben.

\subsection{Netzwerke anlegen}
\label{NetzwerkAnlegen}
Zum anlegen eines neuen Netzwerkes klicken Sie mit der rechten Maustaste auf den
Stammeintrag "`Netzwerke"'. Wählen sie im Kontextmenü den Menüpunkt
"`Neues Netzwerk"'. Bei fokusiertem Stammeintrag können Sie auch die
Tastenkombination \key{Strg+N} nutzen.
Es erscheint die Aufforderung die Netzwerk ID, sowie den Netzwerkpräfix anzugeben.
Alternativ können Sie auch die Netzwerkmaske angeben. Folgende Eingaben sind z.B.
möglich:

\ipAddress{10.5.0.0/24}\\
\ipAddress{192.168.178.0/255.255.255.0}

Achten Sie darauf, dass das eingegebene Netzwerk sich nicht mit bereits vorhanden
überschneidet. In diesem Falle erhalten Sie eine Fehlermeldung.

\subsection{Netzwerk in Subnetze teilen}
Die folgenden Punkte beschreiben Ihre Möglichkeiten Subnetze zu einem Netwerk
hinzuzufügen.

\subsubsection{Einzelnes Subnetz anlegen}
Zum anlegen eines einzelnen Subnetzes wählen sie im Kontextmenü des Netzwerkes den
Menüpunkt "`Neues Subnetz"'. Danach fahren Sie wie in \ref{NetzwerkAnlegen}
beschrieben fort.
Auch hier können Sie bei fokusiertem Netzwerk die Tastenkombination \key{Strg+N}
nutzen.

\subsubsection{Subnetz nach Anzahl der Hosts hinzufügen}
Wenn Sie die Anzahl der zu erwartenen Hosts in einem Subnetz kennen kann der
\subnetcalc\ ein möglichst kleines Subnetzwerk erstellen und es an der erstmöglichen
Position einfügen. Dafür gehen Sie wie folgt vor:

Wählen Sie im Kontextmenü des Netwerkes den Punkt "`Neues Subnetz nach Größe"' und tragen
Sie die Ihnen bekannte Anzahl an Hosts ein. Das neue Subnetzwerk wird automatisch erstellt 
und eingefügt.

\subsubsection{Netzwerke gleichmäßig in Subnetze teilen}
Sie können ein Netzwerk gleichmäßig in Subnetze teilen. Dafür stehen Ihnen zwei
Verfahren zur Verfügung: nach Größe und nach Anzahl.

\paragraph{Nach Größe}
Um ein Netzwerk nach der Größe der Subnetze zu teilen wählen Sie im Kontextmenü
des Netzwerkes den Menüpunkt "`Gleichmäßig nach Größe"'. Wählen Sie nun aus,
wie viele Hosts die Subnetze beinhalten sollen.

\paragraph{Nach Anzahl}
Um ein Netzwerk nach der Anzahl der Subnetze zu teilen wählen Sie im Kontextmenü
des Netzwerkes den Menüpunkt "`Gleichmäßig nach Anzahl"'. Wählen Sie nun aus, wie
in viele Subnetze das Netzwerk geteilt werden soll.

\textbf{Beachten Sie: Bei der gleichmäßigen Unterteilung eines Subnetzes werden alle
vorhandenen Subnetze überschrieben.}

\subsection{Netzwerk entfernen}
Um ein Netzwerk zu entfernen, wählen sie im Kontextmenü den Eintrag
"`Netzwerk/Subnetz löschen"', oder drücken Sie bei fokusiertem Netzwerk die
\key{Entfernen} Taste.

\subsection{Netzwerk unbenennen}
Sie können Netzwerken Namen geben. Um einem Netzwerk einen Namen zuzuweisen, oder
es umzubenennen, wählen Sie im Kontextmenü des Netwerkes den Menüpunkt
"`Umbenennen"'.
Wenn Sie den Namen eines Netzwerkes entfernen möchten, verfahren Sie genau so und lassen
Sie das Eingabefeld leer.