\section{Speichern und Laden}
Sie können eine Planung mit allen Netzwerken speichern und später wiederverwenden.
Alternativ können Sie auch einzelne Netzwerke exportieren und diese später in eine
andere Planung importieren.

\subsection{Netzwerke}
Die folgende Schritte beschreiben den Ex- und Import von einzelnen Netzwerken.

\subsubsection{Exportieren}
Um ein Netzwerk zu speichern, wählen Sie den Menüpunkt "`Speichern unter..."' im
Kontextmenü des Netzwerkes aus. Geben Sie einen Dateinamen und Speicherort ein und
speichern sie das Netzwerk.

Die gespeicherte Version eines Netzwerkes enthält alle Subnetzwerke und Hosts.

\subsubsection{Importieren}
Um ein Netzwerk zu importieren, wählen Sie zunächst das Netzwerk zu dem das zu
importierende Netzwerk hinzugefügt werden soll aus. Dabei kann es sich auch um
den Stammeintrag handeln. Nutzen Sie nun den Menüpunkt "`Netzwerk laden"' im
Kontextmenü und wählen Sie die Datei aus.

Stellen Sie sicher, dass das zu importierende Netzwerk in das übergeordnet
Netzwerk passt und es zu keinen Überscheidungen kommt.

\subsection{Planung}
Um Ihre gesamte Planung, inklusive aller Netzwerke abzuspeichern gehen Sie wie folgt
vor:

\subsubsection{Speichern}
Klicken Sie im Menü auf den Menüpunkt "`Datei"' und anschließend auf
"`Speichern unter..."'. Wähle Sie einen Dateinamen und einen Speicherort aus und
speichern Sie die Planung.

Die gespeicherte Verion der Planung enthält alle Netzwerke, ihre
Subnetzwerke und Hosts, sowie die globale IPv6 Konfiguration.

\subsubsection{Öffnen}
Um eine vorher gespeicherte Planung zu öffnen, wählen Sie im Menü den Menüpunkt
"`Datei"' und anschließend "`Öffnen"' aus. Wählen Sie die Datei und öffnen Sie
diese.

\textbf{Beachten Sie, dass beim Öffnen einer Planung die geöffnete Planung verworfen wird.
Speichern Sie diese bei Bedarf zuerst ab.}