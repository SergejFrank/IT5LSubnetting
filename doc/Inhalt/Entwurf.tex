\section{Projektentwurf}
Dieser Abschnitt dokumentiert die Entwurfsphase des Projektes.

\subsection{Komponentenentwurf}
Zunächst entwarfen wir einen Plan bezüglich der einzelnen Komponenten der Anwendung.
Dabei handelt es sich um die Netzwerklogik, sowie die Frontendlogik bzw. das GUI.
Wir definierten für beide Komponenten jeweils ein Java Package um diese logisch zu trennen, was
sich als unentbehrlich herausgestellt hat um eine klare Übersicht über die Programmstruktur zu behalten
und um die Verteilung der Aufgaben sinnvoll innerhalb der Gruppe gliedern zu können.

\subsection{Entwurf der Netzwerklogik}
Zunächst definierten wir eine Klasse um den elementaren Baustein der Anwendungslogik zu speichern;
Die Klasse \Klasse{IPAddress} stellt alle Daten einer IPv4 Adresse zur Verfügung,
wobei sie keine Logik implementiert. Desweiteren entwarfen wir die Klassen \Klasse{Network} und \Klasse{Subnet}.
Als letztes entwarfen wir die Klasse \Klasse{Host} um Endpunkte in den Netzwerken darstellen zu können.
Die drei Klassen sollten alle die Klasse \Klasse{IPAddress} nutzen um verschiedenste Informationen
zu speichern z.B. Netzwerk ID order Subnetmaske.
Als Sammlung verschiedenster Funktionen definierten wir die Klasse \Klasse{NetUtils}.
Dadurch erhalten wir eine bessere Übersicht innerhalb der Klassen, und können diese Funktionen gleichzeitig
besser allen Klassen zur Verfügung stellen.

\subsection{Entwurf der GUI}
Der erste Entwurf der GUI sah ein Tab-Layout vor, wie es in den Beispielen zu sehen ist,
die dem Projektauftrag beiliegen. Dabei sollte sich der Workflow nacheinander durch eine
Übersicht der Netzwerke, über eine Übersicht der jeweiligen Subnetze, bis zu einer der
dazugehörigen Hosts bewegen. Nach einer ersten Einarbeitung in Swings TreeNodes und SplitPanes
wurde entschieden diese Aufteilung der GUI zu verwerfen, und stattdessen eine Komplettübersicht über 
alle Netzwerke, Subnetze und Hosts in einem Fenster mit Baumstruktur und SplitPanes anzuzeigen.