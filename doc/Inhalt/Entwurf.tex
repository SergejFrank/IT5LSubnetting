\section{Projektentwurf}
Dieser Abschnitt dokumentiert die Entwurfsphase des Projektes.

\subsection{Komponentenentwurf}
Zunächst entwarfen wir einen Plan bezüglich der einzelnen Komponenten der Anwendung.
Dabei handelt es sich um die Netzwerklogik, sowie die Frontendlogik bzw. das GUI.
Wir definierten für beide Komponenten jeweils ein Java Package um diese logisch zu trennen.

\subsection{Entwurf der Netzwerklogik}
Zunächst definierten wir eine Klasse um den elementaren Baustein der Anwendungslogik zu speichern;
Die Klasse \Klasse{IPAddress} stellt alle Daten einer IPv4 Adresse zur verfügung,
wobei sie keine Logik implementiert. Desweiteren entwarfen wir die Klassen \Klasse{Network} und \Klasse{Subnet}.
Als letztes entwarfen wir die Klasse \Klasse{Host} um Endpunkte in den Netzwerken darstellen zu können.
Die drei Klassen sollten alle die Klasse \Klasse{IPAddress} nutzen um verschiedenste Informationen
zu speichern z.B. Netzwerk ID order Subnetmaske.
Als Sammlung verschiedenster Funktionen definierten wir die Klasse \Klasse{NetUtils}.

\subsection{Entwurf der GUI}
Der erste Entwurf der GUI sah ein Tab-Layout vor, wie es in den Beispielen zu sehen ist,
die dem Projektauftrag beiliegen. Dabei sollte sich der Workflow nacheinander durch eine
Übersicht der Netzwerke, über eine Übersichte der jeweiligen Subnetze, bis zu einer der
dazugehörigen Hosts bewegen.