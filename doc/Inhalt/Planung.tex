\section{Projektplanung}
Im folgenden zeigen wir auf, wie wir die Durchführung des Projekts geplant haben.

\subsection{Projektphasen}
Die folgenden Projektphasen wurden von uns in erster Hand festgelegt.

\subsubsection{Planungsphase}
Die Planungsphase soll uns einen Überblick über das Projekt, die Ziele und unsere
Durchführung schaffen. Dabei setzen wir zum Großteil auf Brainstormingrunden, die unsere
Planung am effektivsten unterstützen.
In dieser Phasen legen wir die in \ref{sec:Entwicklungsprozess} aufgezeigte Methodik und die in
\ref{sec:Resourcenplanung} dargestellten Resourcen fest.

\subsubsection{Entwurfsphase}
\label{PhasenEntwurf}
In der Entwurfsphase planen wir alle Komponenten der Software. Dabei versuchen wir unter
anderem folgende Fragen zu beantworten:

\textit{Soll die Software modularisiert werden?\\}
\textit{Wenn ja, wie soll die Modularisierung erfolgen?\\}
\textit{Auf welcher Basis soll die Netzwerklogik entstehen?\\}
\textit{Welche Codekonventionen wollen wir als Grundlage nehmen?\\}
\textit{Sollen Tests eingesetzt werden?\\}
\textit{Wenn ja, welche Tests und wie ausgiebig?\\}
\textit{Wie soll unsere Klassenstruktur aussehen?}

Aus dieser Phase entsteht ein Klassendiagramm, welches einen ersten Überblick über unsere
Implementierung darstellt. Die Antworten auf diese Fragen bilden die Grundlage für die
nächste Phase; die Implementierung.

\subsubsection{Implementierungsphase}
In der Implementierungsphase beginnen wir damit die Logik und Benutzeroberfläche zu
entwickeln. Auf Grundlage der in \ref{PhasenEntwurf} festgelegten Klassenstruktur und
Modularisierung können wir in einem agilen Entwicklungsprozess (siehe
\ref{sec:Entwicklungsprozess}) beginnen die einzelnen Funktionen und Komponenten zu
implementieren.

\subsection{Entwicklungsprozess}
\label{sec:Entwicklungsprozess}
Wir haben uns für einen Agilen Entwicklungsprozess entschieden, um auf spontane Ideen und Anregungen
besser eingehen zu können. Außerdem ermöglicht die Agile Entwicklung mehr Freiraum bei der
Entwicklung, was sich positiv auf das Arbeitsklima, sowie unsere persönlichen Lernerfolge auswirkt.
Die agile Softwareentwicklung schützt uns des weiteren auch vor Änderungen im Projektauftrag, die leider
häufiger auftreten, da wir diese in der Regel besser umsetzen können.

\subsection{Resourcenplanung}
\label{sec:Resourcenplanung}
Als Entwicklungsumgebung haben wir uns für IntelliJ des Herstellers JetBrains entschieden.
Als Versionskontrolle setzen wir auf Git mithilfe eines GitHub Repositories.