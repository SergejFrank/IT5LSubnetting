\section{Projektplanung}
Im folgenden zeigen wir auf, wie wir die Durchführung des Projekts geplant haben.

\subsection{Projektphasen}
Die folgenden Projektphasen wurden von uns in erster Hand festgelegt.

\subsubsection{Planungsphase}
Um einen groben Überblick über das Projekt, die Ziele und die Durchführung zu erlangen,
haben wir zunächst eine ausgiebige Brainstormingrunde veranstaltet. Dies führte zu vielen Ideen
und verbesserte unseren Ausblick auf das Projekt sowie unsere Vorstellung unseres Endergebnisses
enorm.
In dieser Phasen legten wir die in \ref{sec:Entwicklungsprozess} aufgezeigte Methodik und die in
\ref{sec:Resourcenplanung} dargestellten Resourcen.

\subsubsection{Entwurfsphase}
In der Entwurfsphase haben wir die Komponeten der Anwendungs geplant und erste Klassendefinitionen
erstellt. Diese halfen uns während der Implementierung, da die wichtigsten Eigenschaften und Methoden
bereits vorgegeben waren.

\subsubsection{Implementierungsphase}
%\subsubsection{Implementierungsphase der Logik}
%In der folgenden Arbeitszeit folgte die Implementierung der Logik. Um einen guten Programmierstil zu wahren
%und eine leichte Pflege im Laufe des Projektes zu ermöglichen wurde die Logik komplett
%unabhängig zur GUI entwickelt.

%\subsubsection{Implementierungsphase des GUI}
%Durch die logische Trennung der Logik und der GUI war es uns möglich 

\subsection{Entwicklungsprozess}
\label{sec:Entwicklungsprozess}
Wir haben uns für einen Agilen Entwicklungsprozess entschieden, um auf spontane Ideen und Anregungen
besser eingehen zu können. Außerdem ermöglicht die Agile Entwicklung mehr Freiraum bei der
Entwicklung, was sich positiv auf das Arbeitsklima, sowie unsere persönlichen Lernerfolge auswirkt.
Die agile Softwareentwicklung schützt uns des weiteren auch vor Änderungen im Projektauftrag, die leider
häufiger auftreten, da wir diese in der Regel besser umsetzen können.

\subsection{Resourcenplanung}
\label{sec:Resourcenplanung}
Als Entwicklungsumgebung haben wir uns für IntelliJ des Herstellers JetBrains entschieden.
Als Versionskontrolle setzen wir auf Git mithilfe eines GitHub Repositories.